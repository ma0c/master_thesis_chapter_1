\documentclass[10pt, a4paper]{article}
\usepackage{lrec}
\usepackage{multibib}
\usepackage{xcolor}
\newcites{languageresource}{Language Resources} 
\usepackage[utf8]{inputenc}
\usepackage[left=2cm,right=2cm,top=2cm,bottom=3cm]{geometry}
\usepackage{url}

%%%%% COMMENTS %%%%%%%
\usepackage{xcolor}
\usepackage{soul}
\newcommand{\macb}{\textcolor{red}}
%%%%% COMMENTS %%%%%%%

% \title{Existing Spoken Open Source Corpora for the Spanish Language}
\title{Latin American Spanish: A Review of Open Source Spoken Corpora}
\name{Mauricio Andrés Collazos Calambás$^{\ast}$, María Andrea Cruz Blandón$^{\dagger}$, \\
 {\bf \large Raúl Ernesto Gutiérrez de Piñerez Reyes$^{\ast}$}}

\address{$^{\ast}$Escuela de Ingeniería en Sistemas y Computación, Universidad del Valle \\ 
Calle 13 \# 100-00, Cali, Colombia \\ 
          {mauricio.collazos,raul.gutierrez}@correounivalle.edu.co \\
$^{\dagger}$Speech and Cognition Research Group, Tampere University \\ 
\textcolor{red} {Address3} \\ 
\textcolor{red}{mail3@Institution2}          
}
% \name{{Mauricio Collazos}^{1} , Maria Andrea Cruz Blandon, Raúl Ernesto Gutiérrez de Piñerez Reyes}
% You will need to add your affiliation (Universidad del Valle, Cali, Colombia)
% The date is not need it in articles, that will be later added with the publication.

% \address{Universidad del Valle \\
%           Cali, Colombia \\
%          mauricio@correounivalle.edu.co}


 % some catchy summary of the document should go here, some key words might be added as well. 
    
    % Questions that should answer the abstract: 
    % What is the bigger, more general field your article relates to?  What is the purpose of your article?  What methodology did you use? What are the key results? What are the practical implications of your research (how can the results be utilised by e.g. practitioners, society or companies)?
    
\abstract{
Research for Speech technologies relies heavily on high quality and high volume language resources designed for specific languages. Open resources of this kind are low in comparison to other closed resources. The creation of high quality and high volume resources for speech recognition and speech synthesis is expensive, and researchers usually need to create custom datasets for their own investigation. For non-English languages, the resources are even scarce and the gap between English and non-English resources is huge. We present a compilation of open-sourced language resources for Speech Technologies and the Spanish Language, describing each resource and proposing possibles applications for each corpus. \\ \newline \Keywords{Latin American Spanish, Open Source Corpora, Spoken Corpora} }

\begin{document}

\maketitleabstract


\section{Introduction}

Speech technologies rely on language resources used for training, where modern models are data-hungry For example, in SoTA automatic speech recognition and speech synthesis systems such as Listen Attend Spell by Google \cite{Chiu2018} were trained using more than 12500 hours of spoken English. Additionally, some models require high-quality annotated data (e.g., TIMIT \cite{TIMIT}, SwitchBoard \cite{Switchboard}); however, manual annotations are typically costly and time-demanding. Despite the collective efforts for reducing the amount of annotated data (e.g., semi-supervised models \cite{AmazonSemiSupervised}, or unsupervised models \cite{ZeroResources}), those approaches require a great amount of unannotated data instead. These drawbacks become more evident for languages different than English. Even for Spanish, one of the most spoken languages in the world \cite{HernndezMena2017}, the resources are limited in size and variety, especially for annotated corpora.

Compared to the English language availability of open resources are limited, examples of heavily used English corpora used as the baseline for research is deeply observed in the literature, corpora like TIMIT \cite{TIMIT}, Switchboard \cite{Switchboard} or Fisher \cite{Fisher}, has provided enough infrastructure to support research that for years and even now guide state of the art development in Speech Recognition and Speech Synthesis. That corpus previously mentioned contains not only phonetic information but dialectical variability, allowing the creation of more robust systems.

Availability of corpora to research in Speech Recognition is a fundamental key in advances in prototypes and new development of algorithms and techniques and comparing resources of the top five spoken languages in the world searching in the two of the recognized catalogs for language resources, it can be observed that the difference in resources is huge. 

Table \ref{tab:resources_by_langauge} shows the number of resources published by the Linguistic Data Consortium LDC and European Language Resources Association ELRA, showing that published English resources have more 5 times more representation than Mandarin and Spanish.


\begin{table}[h]
\caption{ASR Corpora for the top five spoken languages in the world \cite{HernndezMena2017}}
\label{tab:resources_by_langauge}
\begin{tabular}{|l|l|l|l|}
\hline
Rank & Language & LDC & ELRA \\ \hline
1    & Mandarin & 24  & 6    \\ \hline
2    & English  & 116 & 23   \\ \hline
3    & Spanish  & 20  & 20   \\ \hline
4    & Hindi    & 9   & 4    \\ \hline
5    & Arabic   & 10  & 32   \\ \hline
\end{tabular}
\end{table}


This article shows a review of the open-source resources for Speech Technologies for the Spanish language, using criteria of visibility, availability, and impact, understanding that there are other resources crafted in previous research but the visibility, availability, or impact are limited.

\section{Open Resources for the Spanish Language}

Language Resources for the Spanish languages focused on speech technologies with open licenses and openly distributed over the web are mentioned in this section. A summary is presented in Table \ref{tab:open_source_spanish_corpus} sorted by publishing date and showing important features of the dataset as the Licence of distribution, Annotation Level of the text transcription, Lenght of duration, Recording conditions of the recordings, the Spanish Dialect, the source of the speech recording and its publishing date. The following abbreviations are used in Table \ref{tab:open_source_spanish_corpus}: CC BY-SA 4.0 Creative Commons Share A Like v4.0 license, LDC Non-Members LDC User LDC User Agreement for Non-Members, for dialects MX Mexican, ES Peninsular, AR Argentinian, CH Chilenian, PE Peruvian, CO Colombian, PR Puerto Rico, VE Venezuelan.

\begin{table*}[ht]
\caption{List of Open Source Spanish Corpora}
\label{tab:open_source_spanish_corpus}
\begin{tabular}{|l|l|l|l|l|l|l|l|}
\hline
\textbf{Resource name} & \textbf{Licence}  & \multicolumn{1}{|p{1.8cm}|}{\textbf{Annotation Level}} & \textbf{Length} & \multicolumn{1}{|p{2cm}|}{\textbf{Recording Conditions}} & \textbf{Dialects} & \textbf{Source} &  \multicolumn{1}{|p{2cm}|}{\textbf{Publishing Date}} \\ \hline

{DIMEx100}  & \multicolumn{1}{|p{2cm}|}
            {Free for Academic Purposes}        & {Phonetic} & {5h} & {Studio} & {MX}                       & {Internet} & 2004 \\ \hline
{Heroico}  & \multicolumn{1}{|p{2cm}|}
{LDC Non-Members}              & {Uterance} & {13h}  & {Noisy}  & {MX}                  & \multicolumn{1}{|p{2cm}|}
                                                                                                                 {High School Academic Textbooks} &                                                         2006\\ \hline 

{CIEMPIESS}   & \multicolumn{1}{|p{2cm}|}
             {CC BY-SA 4.0}            & {Word}     & {17h}  & {Studio} & {MX}                 & {Radio Program} & 2014 \\ \hline
\multicolumn{1}{|p{2cm}|}
{CIEMPIESS Light}   & \multicolumn{1}{|p{1.5cm}|}
             {CC BY-SA 4.0}            & {Word}     & {18h}  & {Studio} & {MX}                 & {Radio Program} & 2016 \\ \hline
\multicolumn{1}{|p{2cm}|}
{CIEMPIESS Balance}   & \multicolumn{1}{|p{1.5cm}|}
             {CC BY-SA 4.0}            & {Word}     & {18h}  & {Studio} & {MX}                 & {Radio Program} & 2018 \\ \hline
\multicolumn{1}{|p{2cm}|}
{CIEMPIESS EXP. - Complementary}   & \multicolumn{1}{|p{1.5cm}|}
             {CC BY-SA 4.0}            & {Word}     & {1h}  & {Studio} & {MX}                 & {Radio Program} & 2018 \\ \hline
\multicolumn{1}{|p{2cm}|}
{CIEMPIESS EXP. - FEM}   & \multicolumn{1}{|p{1.5cm}|}
             {CC BY-SA 4.0}            & {Word}     & {13h}  & {Studio} & {MX}                 & {Radio Program} & 2018 \\ \hline
\multicolumn{1}{|p{2cm}|}
{CIEMPIESS EXP. - Test}   & \multicolumn{1}{|p{1.5cm}|}
             {CC BY-SA 4.0}            & {Word}     & {8h}  & {Studio} & {MX}                 & {Radio Program} & 2018 \\ \hline

{M-AILABS}  & \multicolumn{1}{|p{1.5cm}|}
            {M-AILABS BSD License}               & {Uterance} & {108h} & {Noisy}  & \multicolumn{1}{|p{1.5cm}|}
            {ES, MX, AR, Others} 
                                                                                                                & {LibriVox} & 2019 \\ \hline
\multicolumn{1}{|p{2cm}|}
{Google Language 
Resources Latin American} & \multicolumn{1}{|p{1.5cm}|}
                        {CC BY-SA 4.0}  & {Uterance} & {37h}  & {Studio} & \multicolumn{1}{|p{1.5cm}|}
            {AR, CH, CO, PE, PR, VE}       
                                                                                                                &{Internet}   & 2020\\ \hline
                                                                                                    


{Common Voice}  & {CC-0}                            & {Uterance} & {529h} & {Noisy}  &  \multicolumn{1}{|p{1.5cm}|}
            {ES, MX, AR, Others}                                                          & {Internet} & 2020 \\ \hline                

{Librivox Spanish}  & \multicolumn{1}{|p{1.5cm}|}
{LDC Non-Members}               & {Uterance} & {73h} & {Noisy}  &  \multicolumn{1}{|p{1.5cm}|}
            {ES, MX, AR, Others}                                                          & {Internet} & 2020 \\   \hline     
\end{tabular}
\end{table*}

\subsection{DIMEx100}

DIMEx100 is a spoken corpus designed and recorded by the Instituto de Investigaciones en Matematicas Aplicadas y en Sistemas IIMAS at Universidad Nacional Autonoma de Mexico UNAM in 2004, using texts from the Corpus 230\cite{Corpus230}, a phonetically balanced written corpus for the Spanish language. This project was annotated using a newly phonetic dictionary for Mexican Spanish called MEXBET, which improves the representation for Mexican Spanish allophones in comparison from other multilingual phonetic alphabets as SAMPA and WordNET \cite{mexbet}. The recording process for DIMEx100 was performed in a studio recording, with low noise, with the microphone located at a homogeneous distance from the speakers. 100 speakers were selected between 16 and 36 years of age being 50 male and 50 female. The mean age of the speakers was 23 years and all speakers speak the Mexican dialect.

This is a small phonetically and gender-balanced corpus with a small vocabulary phonetically annotated, that can be used for isolated word recognition or bootstrapping for larger resources, however, all speakers use the Mexican Dialect making it slanted for other dialects

\subsection{Heroico}

Heroico is a spoken corpus published in 2006 by John Morgan in the Department of Foreign Languages (DFL) and Center for Technology Enhanced Language Learning (CTELL), all recordings were recorded at the El Heroico Colegio Militar and the Mexican Military Academy in Mexico. This corpus is gender unbalanced and is annotated at the utterance level. HEROICO corpus is composed of 102 speakers answering openly a set of 143 questions, and reading 75 utterances from a dataset composed of 205 short phrases and 519 simple sentences from high school academic notes.

Distribution of Heroico corpus also includes the USMA subcorpus, recorded in 1997 at which is composed of 206 fixed utterances read by 18 different speakers native and non-native. The total duration of the corpus is 13 hours. All recordings were made using desktop computers, noise is present in several recordings \cite{heroico}.

Heroico and USMA corpora combined has 13 hours of audio, and part of the corpora is spontaneous speech in noisy environments and different signal powers which allows this corpus to train models for robust speech recognition. Also, some speakers in the USMA corpus aren't native Spanish speakers and there are different Spanish dialects which also gives variability for speaker-independent models.
 

\subsection{CIEMPIESS Project}

The Corpus de Investigación en Español de México del Posgrado de Ingeniería Eléctrica y Servicio Social from the Universidad Nacional Autonoma de Mexico CIEMPIES-UNAM is a project started in 2012 aiming to develop and share free and open-source tools for speech processing in the Spanish Language project isn't limited to corpora creation but to research in speech technologies applied to the Spanish Language \cite{CIEMPIESS-Webpage}. For corpora annotation, the CIEMPIESS project uses undergraduate students to perform manual annotation, as part of a requirement to receive a diploma. All segmentation and annotation are performed using open source tools and final corpus distributed under open licenses. This project has published several corpus over the years, including the original CIEMPIESS corpus \cite{CIEMPIESS}, CIEMPIESS Light \cite{CIEMPIESS-LIGHT}, CIEMPIESS Balance \cite{CIEMPIESS-BALANCE}, CIEMPIESS Experimentation \cite{CIEMPIESS-Experimentation} and LibriVoxSpanish \cite{LibriVox-Spanish} which will be described below.

\subsubsection{CIEMPIESS}

The Corpus de Investigación en Español de México del Posgrado de Ingeniería Eléctrica y Servicio Social was published in 2014 by the Departamento de Procesamiento Digital de Señales at the Universidad Nacional Autónoma de México, aiming to create a spoken corpus for automatic speech recognition. All recordings were extracted from various radio programs emitted by the University, in total 43 one-hour programs were recorded, manually segmented and only recordings were of a single speaker and no other background sounds as music or interference were selected. Giving the radio program nature, this corpus contains continuous speech and is gender unbalanced, with 77.86 male recordings and only 22.14 female recordings. The full corpus has 16717 recordings as has a total length of 17 hours. The annotation process was done using MEXBET and tonic vowels, also all utterances are aligned by word.

\subsubsection{CIEMPIESS Light}

CIEMPIES Light was published after two years of experimentation with the CIEMPIESS Corpus, a lot of feedback was received by the CIEMPIESS team and a new version of the original corpus was released, changing the distribution format from SPH files to WAV files, also modifying some recordings and adding almost a new hour of audio. This new distribution is compatible with modern ASR tools like Kaldi and CMU Sphinx. Most of the original recordings were kept but others were replaced. This corpus has almost 1 hour plus of recordings.

This corpus can be considered as version two of the original CIEMPIESS corpus, but improved to work with modern ASR systems; and without using any phonetic transcription, also recordings are discriminated by genre and speaker. This corpus includes recordings from 53 male speakers and 34 female speakers. Each speaker is identified by a prefix indicating the genre M for male and F for female and a consecutive number.

\subsubsection{CIEMPIESS Balance}

Since the CIEMPIESS corpus is gender unbalanced, a newly created corpus from the same source was created to balance the CIEMPIESS LIGHT corpus to have together a combined gender-balanced corpus. This corpus is composed of 18 hours and 20 minutes where 12 hours and 40 seconds are from female speakers and 5 hours and 40 minutes are from male speakers.

Distribution of this corpus is similar to the CIEMPIES Light corpus, where recordings are stored in WAV format and discriminated by genre and, as this corpus is a counterpart for the CIEMPIESS Light corpus, the CIEMPIES Balance corpus has 53 Female Speakers and 34 Male speakers, identified by a prefix indicating the genre M for male and F for female and a consecutive number. The recording duration length for each speaker in the CIEMPIESS Light corpus is similar to the counterpart of the opposite gender in the CIEMPIESS Balance Corpus.

The usage of this corpus is encouraged to be used together with the CIEMPIESS Light corpus to have a longer corpus gender balanced of almost 36 hours of continuous speech.

\subsubsection{CIEMPIESS Experimentation}

CIEMPIES Experimentation is a set of three corpora distributed a single one, with a corpus for phonetically balance allophones for the Mexican Spanish, a corpus with only female speakers, and one designed to be a standard test corpus.

CIEMPIESS EXPERIMENTATION - COMPLEMENTARY is a phonetically balanced corpus for isolated words containing one hour of audio annotated using MEXBET 29 and MEXBET 66 two phonetic annotations that consider different allophones for the Spanish Language. This corpus contains recordings from 10 male speakers and 10 female speakers and was created to improve speech recognition engines that didn't find occurrences of specific allophones when training acoustic models.

CIEMPIESS EXPERIMENTATION - FEM is a corpus with only female speakers, containing 13 hours and 54 minutes of recordings, 16 of the female speakers in the corpus are native Mexican speakers and 5 more are other Spanish dialects, including Venezuelan, Argentinian, Spanish from El Salvador, Spanish from Dominican republic and other dialects classified as unknown.

CIEMPIES EXPERIMENTATION - TEST is a gender-balanced corpus designed to test speech applications, having a total of 8 hours and 8 minutes from recordings by 10 female speakers and 10 male speakers 4 hours and 4 minutes each.

\subsubsection{Librivox Spanish}

Librivox Spanish is a Spanish corpus based on open recordings uploaded to the Librivox site \cite{LibriVox}, all audiobooks in the public domain and part of the Guttenberg Project \cite{gutenberg} or released to the public domain. Corpus was annotated manually by undergraduate students as part of their requirement of social service at the Universidad Nacional Autonoma de Mexico, has a total of 73 hours of audio, gender-balanced with 60 hours from native Spanish speakers and the rest from non-native speakers. As this corpus is crowdsourced, recordings have different quality, in most cases, recordings were made in a quiet environment and using regular computer microphones.

As this corpus is manually annotated and original audio recordings are in the public domain, the corpus itself can be used as a testing corpus for automatic segmentation for the original recordings. Also, the corpus has speakers with different dialects which makes it appropriate to create speaker-independent speech recognition systems.


\subsection{M-AILABS}

The M-AILABS Speech Dataset is a spoken corpus created by Imdat Solak using openly available resources from LibriVox and Project Guttenberg. The corpus was multilingual, including German, US English, and British English, Spanish, Italian, Ukrainian Russian, French and Polish.

Spanish subcorpus is 108 hours long and recorded split three subcorpus, a 10 hours female corpus, recorded by a female Mexican speaker a 72 hours male corpus, recorded by an Argentinian and a Spanish speaker, and  25 hours mixed corpus, recorded by several unidentified speakers. All recordings included in this corpus also belong to the LibriVox project.

To annotate at utterance level the source Librivox recordings an automatic script to segment chapters were used, then a Speech To Text online service was used to give an annotation for the signal, later source text was compared with annotation for the signal provided by the online service and a manual verification were performed to discard the audio or adapt the annotation to the signal.

M-AILABS corpus also have different dialects for Spanish Language and was originally designed to improve Automatic Speech Recognition Systems \cite{M-AILABS}.

\subsection{Google TTS Latin American}

The Crowdsourcing Latin American Spanish for Low-Resource Text-to-Speech corpus was created by Google Research and Laboratoire de Sciences Cognitives et Psycholinguistique and Graduate School of Engineering, The University of Tokyo. The corpus was designed to have a high-quality multi-dialect corpus for Latin American text to speech systems. This corpus includes 6 subcorpus for Argentinian, Chilean, Colombian, Peruvian, Puerto Rican, and Venezuelan dialects, and a total of 174 Speakers and 37.7 hours of audio. Sentences were selected based on a conversational system for Mexican Spanish but later was adapted removing any Mexican Spanish specific sentences. Sentences were adapted to each particular dialect and only 30 were kept as canonical.

Recordings were made in a portable vocal booth with a condenser microphone in a close to silence environment. This corpus was designed to create Speech Synthesis Systems for various Latin American Spanish Dialects \cite{googleTTSLatinAmericanSpanishCorpus}.


\subsection{Common Voice}

Mozilla software foundation, created in mid-2017 a crowdsourcing voice collecting platform to gather speech resources to create models for its Deep Speech Project, based on the Baidu Deep Speech \cite{deepspeeh} proposal. Using a web platform, users record short phrases, and also validate other recordings, having an increasing amount of recordings. The last release was 5.1 and included  54 languages, including English, Kinyarwanda, German, French, Catalan, and Cabilio with more than 500 hours, Spanish, Persian Italian, Russian, Polish with over 100 hours, and the remaining languages under 100 hours.

The Spanish dataset is composed of 521 hours of recordings and 290 validated recordings.

As the corpus is growing day by day, and validations are performed by the same community, errors may be present on the corpora, and different speech conditions as well. The idea of this corpus is to create a large vocabulary, multiple speaker corpora to feed data-hungry models for speech recognition \cite{Common-Voice}.

\section{Applications}

Open source corpora mentioned in the previous chapter for the Spanish Language provides resources to work in speech technologies, some of the corpora designed with specific usage from design and others.

\section{Conclusions}

Annotated spoken corpora are an integral part of the Speech Technologies research.

Corpus creation for the Spanish language has gained popularity in the last decade, having the most amount of data published in the last years, showing that is still an open research question for investigators.

Corpus design for specific tasks is very valuable for speech research.

Dialectical segmentation and dialectical balance are important features to consider when creating spoken corpora.

Gathering quality signals for corpus creation is expensive and the scientific community is using crowdsourcing to create and validate large amounts of data to be used in research.

\section{References}

\bibliography{references_chapter_1.bib}
\bibliographystyle{lrec}

\end{document}
