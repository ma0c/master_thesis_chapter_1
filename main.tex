\documentclass[10pt, a4paper]{article}
\usepackage{lrec}
\usepackage{multibib}
\usepackage{xcolor}
\newcites{languageresource}{Language Resources} 
\usepackage[utf8]{inputenc}
\usepackage[left=2cm,right=2cm,top=2cm,bottom=3cm]{geometry}

%%%%% COMMENTS %%%%%%%
\usepackage{xcolor}
\usepackage{soul}
\newcommand{\macb}{\textcolor{red}}
%%%%% COMMENTS %%%%%%%

% \title{Existing Spoken Open Source Corpora for the Spanish Language}
\title{Latin American Spanish: A Review of Open Source Spoken Corpora}
\name{Mauricio Andrés Collazos Calambás$^{\ast}$, María Andrea Cruz Blandón$^{\dagger}$, \\
 {\bf \large Raúl Ernesto Gutiérrez de Piñerez Reyes$^{\ast}$}}

\address{$^{\ast}$Escuela de Ingeniería en Sistemas y Computación, Universidad del Valle \\ 
Calle 13 \# 100-00, Cali, Colombia \\ 
          {mauricio.collazos,raul.gutierrez}@correounivalle.edu.co \\
$^{\dagger}$Speech and Cognition Research Group, Tampere University \\ 
\textcolor{red} {Address3} \\ 
\textcolor{red}{mail3@Institution2}          
}
% \name{{Mauricio Collazos}^{1} , Maria Andrea Cruz Blandon, Raúl Ernesto Gutiérrez de Piñerez Reyes}
% You will need to add your affiliation (Universidad del Valle, Cali, Colombia)
% The date is not need it in articles, that will be later added with the publication.

% \address{Universidad del Valle \\
%           Cali, Colombia \\
%          mauricio@correounivalle.edu.co}


 % some catchy summary of the document should go here, some key words might be added as well. 
    
    % Questions that should answer the abstract: 
    % What is the bigger, more general field your article relates to?  What is the purpose of your article?  What methodology did you use? What are the key results? What are the practical implications of your research (how can the results be utilised by e.g. practitioners, society or companies)?
    
\abstract{
Research for Speech techonogies relies heavily in high quality and high volume language resources designed for specific languages. Open resources of this kind are low in comparison to other closed resources. The creation of high quality and high volume resources for speech recognition and speech synthesis is expensive, and researchers usually need to create custom datasets for their own investigation. For non-English languages, the resources are even scarced and the gap between English and non-English resources is huge. We present a compilation of open sourced language resources for Speech Tecnologies and the Spanish Language, describing each resource and proposing possibles applications for each corpus. \\ \newline \Keywords{Latin American Spanish, Open Source Corpora, Spoken Corpora} }

\begin{document}

\maketitleabstract


\section{Introduction}

Speech technologies relies on language resources used for training, where modern models are data-hungry For example, in SoTA automatic speech recognition and speech synthesis  systems such as Listen Attend Speell by Google \cite{Chiu2018} were trained using more than 12500 hours of spoken English. Additionally, some models require high-quality annotated data (e.g., TIMIT \cite{TIMIT}, SwitchBoard \cite{Switchboard}); however, manual annotations are typically costly and time-demanding. Despite the collective efforts for reducing the amount of annotated data (e.g., semi-supervised models \cite{AmazonSemiSupervised}, or unsupervised models \cite{ZeroResources}), those approaches require great amount of unannotated data instead. These drawbacks become more evident for languages different than English. Even for Spanish, one of the most spoken language in the world \cite{HernndezMena2017}, the resources are limited in size and variety, especially for annotated corpora.

Compared to the English language availability of open resurces are limited, examples of heavily used English corpora used as base line for research is deeply observed in literature, corpora like TIMIT \cite{TIMIT}, Switchboard \cite{Switchboard} or Fisher \cite{Fisher}, has provide enough infraestructure to support research that for years and even now guide state of the art development in Speech Recognition and Speech Synthesis. Those corpus previusly mentioned contains not only phonetic information but dialectical variability, allowing to create more robust systems.

Availability of corpora to research in Speech Recognition is a fundamental key in advances in prototypes and new development of algorithms and techniques and comparing resources of the top five spoken languages in the world searching in the two of the recognized catalogs for language resources, it can be observed that the difference in resources is huge. 

Table \ref{tab:resources_by_langauge} shows the amount of resources published by the Linguistic Data Consortium LDC and European Language Resources Association ELRA, showing that published English resources has more 5 times more representation than Mandarin and Spanish.


\begin{table}[h]
\caption{ASR Corpora for the top five spoken languages in the world \cite{HernndezMena2017}}
\label{tab:resources_by_langauge}
\begin{tabular}{llll}
Rank & Language & LDC & ELRA \\
1    & Mandarin & 24  & 6    \\
2    & English  & 116 & 23   \\
3    & Spanish  & 20  & 20   \\
4    & Hindi    & 9   & 4    \\
5    & Arabic   & 10  & 32  
\end{tabular}
\end{table}


In this article we show a reeview of the open source resources for Speech Technologies for the Spanish language, using criteria of visibility, availability and impact, understanding that there are other resources crafted in previous research but the visibility, availability or impact are limited.

\section{Open Resources for the Spanish Language}

Language Resources for the Spanish languages focused on speech technologies with open licenses and openly distributed over the web are mentioned in this section. A summary is presented in Table \ref{tab:open_source_spanish_corpus} sorted by publishing date and showing important features of the dataset as the Licence of distribution, Annotation Level of the text transcription, Lenght of duration, Recording conditions of the recordings, the Spanish Dialect, the source of the speech recording and its publising date.

\begin{table*}[ht]
\caption{List of Open Source Spanish Corpora}
\label{tab:open_source_spanish_corpus}
\begin{tabular}{llllllll}
\textbf{Resource name} & \textbf{Licence}  & \multicolumn{1}{p{2cm}}{\textbf{Annotation Level}} & \textbf{Length} & \multicolumn{1}{p{2cm}}{\textbf{Recording Conditions}} & \textbf{Dialect} & \textbf{Source} &  \multicolumn{1}{p{2cm}}{\textbf{Publishing Date}}\\

{DIMEx100}  & \multicolumn{1}{p{2cm}}
            {Free for Academic Purposes}        & {Phonetic} & {5h} & {Studio} & {Mexican}                       & {Internet} & 2004 \\
{Heroico}  & \multicolumn{1}{p{2cm}}
{LDC User Agreement for Non-Members}              & {Uterance} & {13h}  & {Noisy}  & {Mexican}                  & \multicolumn{1}{p{2cm}}
                                                                                                                 {High School Academic Textbooks} & 2006\\

{CIEMPIESS}   & \multicolumn{1}{p{2cm}}
             {CC-Share A like v4.0}            & {Word}     & {17h}  & {Studio} & {Mexican}                 & {Radio Program} & 2014 \\
\multicolumn{1}{p{2cm}}
{CIEMPIESS LIGHT}   & \multicolumn{1}{p{2cm}}
             {CC-Share A like v4.0}            & {Word}     & {18h}  & {Studio} & {Mexican}                 & {Radio Program} & 2016 \\
\multicolumn{1}{p{2cm}}
{CIEMPIESS BALANCE}   & \multicolumn{1}{p{2cm}}
             {CC-Share A like v4.0}            & {Word}     & {18h}  & {Studio} & {Mexican}                 & {Radio Program} & 2018 \\
\multicolumn{1}{p{2cm}}
{CIEMPIESS EXP. - COMPLEMENTARY}   & \multicolumn{1}{p{2cm}}
             {CC-Share A like v4.0}            & {Word}     & {1h}  & {Studio} & {Mexican}                 & {Radio Program} & 2018 \\
\multicolumn{1}{p{2cm}}
{CIEMPIESS EXP. - FEM}   & \multicolumn{1}{p{2cm}}
             {CC-Share A like v4.0}            & {Word}     & {13h}  & {Studio} & {Mexican}                 & {Radio Program} & 2018 \\
\multicolumn{1}{p{2cm}}
{CIEMPIESS EXP. - TEST}   & \multicolumn{1}{p{2cm}}
             {CC-Share A like v4.0}            & {Word}     & {8h}  & {Studio} & {Mexican}                 & {Radio Program} & 2018 \\

{M-AILABS}  & \multicolumn{1}{p{2cm}}
            {M-AILABS BSD License}               & {Uterance} & {108h} & {Noisy}  & \multicolumn{1}{p{2cm}}
            {Peninsular, Mexican, Argentinian, Others} 
                                                                                                                & {LibriVox} & 2019\\
\multicolumn{1}{p{2cm}}
{Google Language 
Resources Latin American} & \multicolumn{1}{p{2cm}}
                        {CC-Share A like v4.0}  & {Uterance} & {37h}  & {Studio} & \multicolumn{1}{p{2cm}}
            {Argentinean, Chilean, Colombian, Peruvian, Puerto Rican, Venezuelan}       
                                                                                                                &{Internet}   & 2020\\
                                                                                                    


{Common Voice}  & {CC-0}                            & {Uterance} & {529h} & {Noisy}  &  \multicolumn{1}{p{2cm}}
            {Peninsular, Mexican, Argentinian, Others}                                                          & {Internet} & 2020\\                 

{Librivox Spanish}  & \multicolumn{1}{p{2cm}}
{LDC User Agreement for Non-Members}               & {Uterance} & {73h} & {Noisy}  &  \multicolumn{1}{p{2cm}}
            {Peninsular, Mexican, Argentinian, Others}                                                          & {Internet} & 2020       
\end{tabular}
\end{table*}

\subsection{DIMEx100}

DIMEx100 is a spoken corpora designed and recorded  by the Instituto de Investigaciones en Matematicas Aplicadas y en Sistemas IIMAS at Universidad Nacional Autonoma de Mexico UNAM in 2004, using texts from the Corpus 230\cite{Corpus230}, a phonetically balanced written corpus for the Spanish language. 

This project is annotated using a newly phonetic dictionary for the Mexican Spanish called MEXBET, which improves the representation for Mexican Spanish allophones in comparison from other multilingual phonetic alphabets as SAMPA and WordNET \cite{mexbet}. 

The recording process for DIMEx100 was performed in a studio recording, with low noise, where the microphone was placed in a homogeneous distance from each speaker, guaranteeing similarity in the power of the signal. As this corpus is gender balanced, from the 100 speakers 50 were male and 50 female, between 16 and 36 years with a mean age of 23 years. All speakers are native Spanish speakers and have an Mexican accent.

\subsection{CIEMPIESS}

Che Corpus de Investigación en Español de México del Posgrado de Ingeniería Eléctrica y Servicio Social was published in 2014 by  the Departamento de PRocesamiento Digital de Señales at the Universidad Nacional Autónoma de México, aiming to create a spoken corpora for automatic speech recognition. All recordings were extracted from various radio programs emitted by the University, in total 43 one-hour programs were recorded, manually segmented and only recordings were of a single speaker and no other background sounds as music or interference were selected.

Giving the radio program nature, this corpus contains continous speech and is gender unbalanced, with 77.86 male recordings and only 22.14 felame recordings. Full corpus has 16717 recordings as has a total lenght of 17 hours.

Annotation process was done using MEXBET and tonic vowels, also all uterances are aligned by word

\subsubsection{CIEMPIESS LIGHT}

After two years of experimentation with the CIEMPIESS Corpus, a lot of feedback was received by the CIEMPIESS team and a newly version of the original corpus was released, changing the distribution format from SPH files to WAV files, also modifying some recordings and adding almost a new hour of audio. This new distribution is compatible with modern ASR tools like Kaldi and CMU Sphinx

\subsubsection{CIEMPIESS BALANCE}

Since the CIEMPIESS corpus is gender unbalanced, a newly created corpus from the same source was created to balance the CIEMPIESS LIGHT corpus to have together a combined gender balanced corpus. This corpus is composed by 18 hours and 20 minutes where 12 hours and 40 seconds are from female speakers and 5 hours and 40 minutes are from male speakers

\subsubsection{CIEMPIESS EXPERIMENTATION}

CIEMPIES EXPERIMENTATION are a set of three corpora designed for specific reasons, CIEMPIESS EXPERIMENTATION - COMPLEMENTARY is a phonetically balanced corpus for isolated words containing one hour of audio annotated using MEXBET

CIEMPIESS EXPERIMENTATION - FEM is a corpus with onle female speakers, containing 13 hours and 54 minutes of recordings

CIEMPIES EXPERIMENTATION - TEST is a gender balanced corpus designed to test speech applications, having a total of 8 hours and 8 minutes from recorddings by 10 felame speakers and 10 male speakers 4 hours and 4 minutes each

\subsection{Heroico}

Heroico is a spoken corpus published in 2006 by John Morgan in the Department of Foreign Languages (DFL) and Center for Technology Enhanced Language Learning (CTELL), all recordings were recorded at the El Heroico Colegio Militar and the Mexican Military Academy in Mexico. This corpus is gender un balanced and is annotated at the utterance leve.


HEROICO corpus is composed of 102 speakers answering openly a set of 143 questions and also reading 75 utterances from an utterance dataset composed for 205 short phrases and 519 simple sentences from lecture notes.

Distribution of Heroico corpus also include the USMA subcorpus, recorded in 1997 at which is composed from 206 fixed utterances read by 18 different speakers native and non-native. Total duration of the corpus is 13 hours. All recordings were made a using desktop computers, noise is present in several recordings \cite{heroico}.
 
\subsection{Google TTS Latin American}

The Crowdsourcing Latin American Spanish for Low-Resource Text-to-Speech corpus was created by Google Research and Laboratoire de Sciences Cognitives et Psycholinguistique and Graduate School of Engineering, The University of Tokyo. The corpus was designed to have a high quality multi-dialect corpora for latin american text to speech systems.

This corpus includes 6 sub corpus for Argentinian, Chilean, Colombian, Peruvian, Puerto Rican and Venezuelan dialects, and a total of 174 Speakers and 37.7 hours of audio.

Sentences were selected based on a conversational system for Mexican Spanish but later was adapted removing any Mexican Spanish specific sentences. Sentences were adapted to each particular dialect and only 30 were kept as canonical.

Recordings were made in a portable vocal booth with a condenser microphone \cite{googleTTSLatinAmericanSpanishCorpus}

\subsection{M-AILABS}

The M-AILABS Speech Dataset is a spoken corpus created by Imdat Solak using openly available resources from LibriVox and Project Guttemberg.

The corpus was multilingual, including German, US English and British English, Spanish, Italian, Ukranian Russian, French and Polish.

Spanish subcorpus is 108 hours long and recorded splitted three subcorpus, a 10 hours female corpus, recorded by a female mexican speaker a 72 hours male corpus, recorded by an argentinian and a spanish speaker and  25 hours mixed corpus, recorded by several unidentified speakers

All recordings were captured using desktop devices.

To annotate at utterance level the source librivox recordings an automatic script to segmentate chapters were used, then a Speech To Text online service were used to give an annotation for the signal, later source text were compared with annotation for the signal provide by the online service and a manual verifation were performed to discard the audio or adapt the annotation to the signal.

\subsection{Common Voice}

Mozilla software foundation, created in mid 2017 a crowdsourcing voice collecting platform to gather speech resources to create models for its Deep Speech Project, based on the Baido Deep Speech \cite{deepspeeh} proposal.

Using a web platform, users record short phrases and also validate other recordings, having a increasing amount of recordings. Last release was 5.1 and included  54 languages.

The Spanish dataset is composed of 521 hours of recordings and 290 validated recordings.

Quality of recordings is good, but since different conditions are present on each speaker, recordings aren't homogeneous.

\subsection{Librivox Spanish}

Livrivox Spanish si a spanish corpus based on open recordings uploaded to the Librivox site, all audiobooks in the public domain and part of the Guttenberg Project or released to the public domain.

Corpus was annotated manually by under graduate students as part of their requirement of social service at the Universidad Nacional Autonoma de Mexico, has a total of 73 hours of audio, gender balanced with 60 hours from native spanish speakers
and the rest from non-native speakers.

As this corpus is crowsourced, recordings have different quality, in most cases recordings were made in a quiet environment and using regular computer microphones

\section{Applications}

\section{Conclusions}

\bibliography{references_chapter_1.bib}
\bibliographystyle{plain}

\end{document}
