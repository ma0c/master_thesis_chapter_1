\documentclass{article}
\usepackage[utf8]{inputenc}
\usepackage[left=2cm,right=2cm,top=2cm,bottom=3cm]{geometry}

\title{Existing Spoken Open Source Corpora for the Spanish Language}
\author{Mauricio Collazos}
\date{August 2020}

\begin{document}

\maketitle

\section{Introduction}

Many aspects of speech technologies relies on availability of language resources, for automatic speech recognition and speech synthesis large amounts of annotated data are required to generate acoustic models to recognize and generate voice signals. The Spanish language is one of the top five most spoken languages in the world, behind English and Manadrin, however availability of Open-Sourced speech resources are small compared with other languages like English.

Open-sourced speech resources for the spanish language are studied in this chapter, analysing speakers, lenght, recording conditions, source of language, dialect, format, annotation level and suggested usage.


\section{Preamble}

Table \ref{tab:resources_by_langauge} shows the amount of resources published by the Linguistic Data Consortium LDC and European Language Resources Association ELRA, noting the difference in availability of resourcer for the top 5 spoken languages in the world. This resources are both open sourced and licenced.


\begin{table}[h]
\caption{ASR Corpora for the top five spoken languages in the world \cite{HernndezMena2017}}
\label{tab:resources_by_langauge}
\begin{tabular}{llll}
Rank & Language & LDC & ELRA \\
1    & Mandarin & 24  & 6    \\
2    & English  & 116 & 23   \\
3    & Spanish  & 20  & 20   \\
4    & Hindi    & 9   & 4    \\
5    & Arabic   & 10  & 32  
\end{tabular}
\end{table}



\begin{table}[ht]
\caption{List of Open Source Spanish Corpora}
\label{tab:open_source_spanish_corpus}
\begin{tabular}{llllllll}
\textbf{Resource name} & \textbf{Licence}  & \multicolumn{1}{p{2cm}}{\textbf{Annotation Level}} & \textbf{Length} & \multicolumn{1}{p{2cm}}{\textbf{Recording Conditions}} & \textbf{Accents} & \textbf{Source} &  \multicolumn{1}{p{2cm}}{\textbf{Publishing Date}}\\

{DIMEx100}  & \multicolumn{1}{p{2cm}}
            {Free for Academic Purposes}        & {Phonetic} & {5h} & {Studio} & {Mexican}                       & {Internet} & 2004 \\

{CIEMPIESS}   & \multicolumn{1}{p{2cm}}
             {CC-Share A like v4.0}            & {Word}     & {17h}  & {Studio} & {Mexican}                 & {Radio Program} & 2014 \\

{Heroico}  & \multicolumn{1}{p{2cm}}
{LDC User Agreement for Non-Members}              & {Uterance} & {13h}  & {Noisy}  & {Mexican}                  & \multicolumn{1}{p{2cm}}
                                                                                                                 {High School Academic Textbooks} & 2006\\

\multicolumn{1}{p{2cm}}
{Google Language 
Resources Latin American} & \multicolumn{1}{p{2cm}}
                        {CC-Share A like v4.0}  & {Uterance} & {37h}  & {Studio} & \multicolumn{1}{p{2cm}}
            {Argentinean, Chilean, Colombian, Peruvian, Puerto Rican, Venezuelan}       
                                                                                                                &{Internet}   & 2020\\
                                                                                                    
{M-AILABS}  & \multicolumn{1}{p{2cm}}
            {M-AILABS BSD License}               & {Uterance} & {108h} & {Noisy}  & \multicolumn{1}{p{2cm}}
            {Peninsular, Mexican, Argentinian, Others} 
                                                                                                                & {LibriVox} & 2019\\

{Common Voice}  & {CC-0}                            & {Uterance} & {529h} & {Noisy}  &  \multicolumn{1}{p{2cm}}
            {Peninsular, Mexican, Argentinian, Others}                                                          & {Internet} & 2020                      
\end{tabular}
\end{table}

\bibliography{references_chapter_1.bib}
\bibliographystyle{plain}

\end{document}
