\documentclass{article}
\usepackage[utf8]{inputenc}
\usepackage[left=2cm,right=2cm,top=2cm,bottom=3cm]{geometry}

\title{Existing Spoken Open Source Corpora for the Spanish Language}
\author{Mauricio Collazos}
\date{August 2020}

\begin{document}

\maketitle

\section{Introduction}

Many aspects of speech technologies relies on availability of language resources, for automatic speech recognition and speech synthesis large amounts of annotated data are required to generate acoustic models to recognize and generate voice signals. The Spanish language is one of the top five most spoken languages in the world, behind English and Manadrin, however availability of Open-Sourced speech resources are small compared with other languages like English.

Open-sourced speech resources for the spanish language are studied in this chapter, analysing speakers, lenght, recording conditions, source of language, dialect, format, annotation level and suggested usage.


\section{Preamble}

Table \ref{tab:resources_by_langauge} shows the amount of resources published by the Linguistic Data Consortium LDC and European Language Resources Association ELRA, noting the difference in availability of resourcer for the top 5 spoken languages in the world. This resources are both open sourced and licenced.


\begin{table}[h]
\caption{ASR Corpora for the top five spoken languages in the world \cite{HernndezMena2017}}
\label{tab:resources_by_langauge}
\begin{tabular}{llll}
Rank & Language & LDC & ELRA \\
1    & Mandarin & 24  & 6    \\
2    & English  & 116 & 23   \\
3    & Spanish  & 20  & 20   \\
4    & Hindi    & 9   & 4    \\
5    & Arabic   & 10  & 32  
\end{tabular}
\end{table}

\section{Open Resources for the Spanish Language}

Language Resources for the Spanish languages focused on speech technologies with open licenses and openly distributed over the web are mentioned in this section. 

\subsection{DIMEx100}

DIMEx100 is a spoken corpora designed and recorded  by the Instituto de Investigaciones en Matematicas Aplicadas y en Sistemas IIMAS at Universidad Nacional Autonoma de Mexico UNAM in 2004, using texts from the Corpus 230\cite{Corpus230}, a phonetically balanced written corpus for the Spanish language. 

This project is annotated using a newly phonetic dictionary for the Mexican Spanish called MEXBET, which improves the representation for Mexican Spanish allophones in comparison from other multilingual phonetic alphabets as SAMPA and WordNET \cite{mexbet}. 

The recording process for DIMEx100 was performed in a studio recording, with low noise, where the microphone was placed in a homogeneous distance from each speaker, guaranteeing similarity in the power of the signal. As this corpus is gender balanced, from the 100 speakers 50 were male and 50 female, between 16 and 36 years with a mean age of 23 years. All speakers are native Spanish speakers and have an Mexican accent.

\subsection{CIEMPIESS}

Che Corpus de Investigación en Español de México del Posgrado de Ingeniería Eléctrica y Servicio Social was published in 2014 by  the Departamento de PRocesamiento Digital de Señales at the Universidad Nacional Autónoma de México, aiming to create a spoken corpora for automatic speech recognition. All recordings were extracted from various radio programs emitted by the University, in total 43 one-hour programs were recorded, manually segmented and only recordings were of a single speaker and no other background sounds as music or interference were selected.

Giving the radio program nature, this corpus contains continous speech and is gender unbalanced, with 77.86 male recordings and only 22.14 felame recordings. Full corpus has 16717 recordings as has a total lenght of 17 hours.

Annotation process was done using MEXBET and tonic vowels, also all uterances are aligned by word

\subsection{Heroico}

Heroico is a spoken corpus published in 2006 by John Morgan in the Department of Foreign Languages (DFL) and Center for Technology Enhanced Language Learning (CTELL), all recordings were recorded at the El Heroico Colegio Militar and the Mexican Military Academy in Mexico. In the Corpus data from non-native speakers were recorded.

HEROICO corpus is composed of 102 speakers answering openly a set of 143 questions and also reading 75 utterances from an utterance dataset composed for 205 short phrases and 519 simple sentences from lecture notes.

USMA corpus is composed from 206 fixed utterances read by 18 different speakers native and non-native. Total duration of the corpus is 13 hours. All recordings were made a using desktop computers, noise is present in several recordings.
 
\subsection{Google TTS Latin American}

The Crowdsourcing Latin American Spanish for Low-Resource Text-to-Speech corpus was created by Google Research and Laboratoire de Sciences Cognitives et Psycholinguistique and Graduate School of Engineering, The University of Tokyo.

The corpus was designed to have a high quality multi-dialect corpora for latin american text to speech systems.

The corpus includes 6 sub corpus for Argentinian, Chilean, Colombian, Peruvian, Puerto Rican and Venezuelan dialects, and a total of 174 Speakers and 37.7 hours of audio.

Sentences were selected based on a conversational system for Mexican Spanish but later was adapted removing any Mexican Spanish specific sentences. Sentences were adapted to each particular dialect and only 30 were kept as canonical.

Recordings were made in a portable vocal booth with a condenser microphone

\subsection{M-AILABS}

The M-AILABS Speech Dataset is a spoken corpus created by Imdat Solak using openly available resources from LibriVox and Project Guttemberg.

The corpus was multilingual, including German, US English and British English, Spanish, Italian, Ukranian Russian, French and Polish.

Spanish subcorpus is 108 hours long and recorded splitted three subcorpus, a 10 hours female corpus, recorded by a female mexican speaker a 72 hours male corpus, recorded by an argentinian and a spanish speaker and  25 hours mixed corpus, recorded by several unidentified speakers

All recordings were captured using desktop devices

\subsection{Common Voice}

Mozilla software foundation, created in mid 2017 a crowdsourcing voice collecting platform to gather speech resources to create models for its Deep Speech Project, based on the Baido Deep Speech proposal.

Using a web platform, users record short phrases and also validate other recordings, having a increasing amount of recordings. Last release was 5.1 and included  54 languages.

The Spanish dataset is composed of 521 hours of recordings and 290 validated recordings 

\subsection{Librivox Spanish}

Livrivox Spanish si a spanish corpus based on open recordings uploaded to the Librivox site, all audiobooks in the public domain and part of the Guttenberg Project or released to the public domain.

Corpus was annotated manually by under graduate students as part of their requirement of social service at the Universidad Nacional Autonoma de Mexico, has a total of 73 hours of audio, gender balanced with 60 hours from native spanish speakers
and the rest from non-native speakers.

As this corpus is crowsourced, recordings have different quality, in most cases recordings were made in a quiet environment and using regular computer microphones


\begin{table}[ht]
\caption{List of Open Source Spanish Corpora}
\label{tab:open_source_spanish_corpus}
\begin{tabular}{llllllll}
\textbf{Resource name} & \textbf{Licence}  & \multicolumn{1}{p{2cm}}{\textbf{Annotation Level}} & \textbf{Length} & \multicolumn{1}{p{2cm}}{\textbf{Recording Conditions}} & \textbf{Accents} & \textbf{Source} &  \multicolumn{1}{p{2cm}}{\textbf{Publishing Date}}\\

{DIMEx100}  & \multicolumn{1}{p{2cm}}
            {Free for Academic Purposes}        & {Phonetic} & {5h} & {Studio} & {Mexican}                       & {Internet} & 2004 \\

{CIEMPIESS}   & \multicolumn{1}{p{2cm}}
             {CC-Share A like v4.0}            & {Word}     & {17h}  & {Studio} & {Mexican}                 & {Radio Program} & 2014 \\

{Heroico}  & \multicolumn{1}{p{2cm}}
{LDC User Agreement for Non-Members}              & {Uterance} & {13h}  & {Noisy}  & {Mexican}                  & \multicolumn{1}{p{2cm}}
                                                                                                                 {High School Academic Textbooks} & 2006\\

\multicolumn{1}{p{2cm}}
{Google Language 
Resources Latin American} & \multicolumn{1}{p{2cm}}
                        {CC-Share A like v4.0}  & {Uterance} & {37h}  & {Studio} & \multicolumn{1}{p{2cm}}
            {Argentinean, Chilean, Colombian, Peruvian, Puerto Rican, Venezuelan}       
                                                                                                                &{Internet}   & 2020\\
                                                                                                    
{M-AILABS}  & \multicolumn{1}{p{2cm}}
            {M-AILABS BSD License}               & {Uterance} & {108h} & {Noisy}  & \multicolumn{1}{p{2cm}}
            {Peninsular, Mexican, Argentinian, Others} 
                                                                                                                & {LibriVox} & 2019\\

{Common Voice}  & {CC-0}                            & {Uterance} & {529h} & {Noisy}  &  \multicolumn{1}{p{2cm}}
            {Peninsular, Mexican, Argentinian, Others}                                                          & {Internet} & 2020                      
\end{tabular}
\end{table}

\bibliography{references_chapter_1.bib}
\bibliographystyle{plain}

\end{document}
